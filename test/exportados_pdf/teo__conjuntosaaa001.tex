\documentclass[12pt]{article}
\usepackage{miestilo}
\begin{document}

\section*{Existencia del ínfimo}

\begin{theorem}
Suponga que S es un conjunto ordenado con la propiedad de la menor cota superior, $\emptyset \neq B \subset S$, y $B$ es una cota inferior. 
Sea $L$ el conjunto de todas las cotas inferiores de $B$. Entonces
\[ \alpha = sup\;\; L\]
existe en $S$, $\alpha = inf\;B \in S.$ 
\end{theorem}

\begin{proof}
Dado que $B$ está acotado inferiormente, $L$ es no vacio. Por otro lado, $L$ consiste en en exactamente aquellas $y \in S$ que satisfacen la desigualdad
\[\forall x \in B: y \leq x.\]
Por lo anterior, observamos que  todo elemento en $B$ es una cota superior de $L$. Entonces, $L$ está acotado superiormente. Como $L \subset S$, por
hipótesis $\alpha = sup \;\; L \in S$. 
\begin{enumerate}
\item Si $\gamma \leq \alpha$, entonces $\gamma$ no es una cota superior de $L$ y por lo tanto, $\gamma \notin B$. De ahi que $\forall x \in B: \alpha \leq x$.
Y por definición $\alpha \in B.$
\item Si $\alpha < \beta$, entonces $\beta \notin L$ ya que $\alpha$  es una cota superior de $L$.
\end{enumerate}
En conclusión, $\alpha$ es una cota inferior de $B$ pero todo elemento $\beta > \apha$ no es  cota inferior de $B$. Por lo tanto $\alpha = inf\;\;B$.
\end{proof}

\section*{Referencia}
Rudin, W., 1964, Principles of Mathematical Analysis, Ed. 3 Cap. 1, Pag. 5 

\section*{Notas Zettelkasten}
\begin{itemize}
  \item \textbf{Enlaces Entrada}: def:conjuntosaaa010, def:conjuntosaaa011, def:conjuntosaaa012, -
  \item \textbf{Enlaces Salida}: -, -, -, -, -
  \item \textbf{Inspirado En}: -, --, -
  \item \textbf{Creado A Partir De}: -, -, -, -, -
\end{itemize}
\end{document}