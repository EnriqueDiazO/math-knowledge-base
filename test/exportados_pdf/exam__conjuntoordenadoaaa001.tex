\documentclass[12pt]{article}
\usepackage{miestilo}
\begin{document}

\section*{Números racionales}

\begin{example}
Los números racionales $\mathbb{Q}$ es un conjunto ordenado si $r < s$ se define como $s - r$ es un número racional positivo.
\end{example}

\section*{Referencia}
Rudin, W., 1964, Principles of Mathematical Analysis, Ed. 3 Cap. 1, Pag. 3 

\section*{Notas Zettelkasten}
\begin{itemize}
  \item \textbf{Enlaces Entrada}: def:conjuntosaaa007, -, -, -, -
  \item \textbf{Enlaces Salida}: -, -, -, -, -
  \item \textbf{Inspirado En}: -, --, -
  \item \textbf{Creado A Partir De}: -, -, -, -, -
\end{itemize}
\end{document}