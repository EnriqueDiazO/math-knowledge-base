\documentclass[12pt]{article}
\usepackage{miestilo}
\begin{document}

\section*{Orden en un conjunto}

\begin{definition}
Sea $S$ un conjunto. Un orden en $S$ es una relación , denotada por $<$ con las siguientes dos propiedades:
\begin{enumerate}
\item Si $x \in S \land y \in S$, entonces una y solo una de las siguientes afirmaciones es cierta
\[x < y, \;\; x = y, \;\; y < x \]
\item Si $x, y, z \in S$ y $x < y,$ e  $y < z$, entonces $x < z$.
\end{enumerate}
A la afirmación $x < y$ puede leerse de maneras diferentes: 
\begin{itemize}
\item $x$ es menor que $y$.
\item $x$ es mas pequeño que $y$.
\item $x$ precede a $y$.
\end{itemize}
\end{definition}

\section*{Referencia}
Rudin, W., 1964, Principles of Mathematical Analysis, Ed. 3 Cap. 1, Pag. 3 

\section*{Notas Zettelkasten}
\begin{itemize}
  \item \textbf{Enlaces Entrada}: def:conjuntosaaa001, -, -, -, -
  \item \textbf{Enlaces Salida}: -, -, -, -, -
  \item \textbf{Inspirado En}: -, --, -
  \item \textbf{Creado A Partir De}: -, -, -, -, -
\end{itemize}
\end{document}