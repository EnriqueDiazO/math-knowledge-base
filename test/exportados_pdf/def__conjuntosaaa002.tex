\documentclass[12pt]{article}
\usepackage{miestilo}
\begin{document}

\section*{Uniones e intersecciones arbitrarias}

\begin{definition}
Denotaremos  con $ A \cup B$   y $A \cap B$ como la unión y la intersección de $A$ y $B$ respectivamente. Luego, si $\{ A_{\alpha}\}$ es una colección de conjuntos, donde $\alpha$ toma valores en algún conjunto índice $I$, escribiremos
\begin{enumerate}
\item $\bigcup_{\alpha \in I} A_{\alpha}$ para la unión de conjuntos de $\{ A_{\alpha}\}$, donde
\[ \bigcup_{\alpha \in I} A_{\alpha} := \{ x :\exists \alpha \in I: x \in A_{\alpha} \} \]
\item $\bigcap_{\alpha \in I} A_{\alpha}$ para la intersección de conjuntos de $\{ A_{\alpha}\}$.
\[ \bigcap_{\alpha \in I} A_{\alpha} := \{ x :\forall \alpha \in I: x \in A_{\alpha} \} \]
\item  Si $I$ es el conjunto de los enteros positivos, entonces
\[ \bigcup_{\alpha \in I} A_{\alpha} = \bigcup_{i=1}^{\infty} A_{\alpha} \]
\[ \bigcap_{\alpha \in I} A_{\alpha} = \bigcap_{i=1}^{\infty} A_{\alpha} \]
\end{enumerate}
\end{definition}

\section*{Referencia}
Rudin, W., 1987, Real and Complex Analysis, Ed. 3 Cap. 1, Pag. 6 

\section*{Notas Zettelkasten}
\begin{itemize}
  \item \textbf{Enlaces Entrada}: def:conjuntosaaa001, -, -, -, -
  \item \textbf{Enlaces Salida}: -, -, -, -, -
  \item \textbf{Inspirado En}: -, --, -
  \item \textbf{Creado A Partir De}: -, -, -, -, -
\end{itemize}
\end{document}