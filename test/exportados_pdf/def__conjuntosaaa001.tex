\documentclass[12pt]{article}
\usepackage{miestilo}
\begin{document}

\section*{Conjunto}

\begin{definition}
Denotaremos  $\{x : P \}$ al conjunto de todos los elementos que satisfacen la propiedad $P$. El símbolo $\emptyset$ se usará para el conjunto que no tiene ningún elemento. Por otro lado, las palabras colección, familia y clase serán usadas como sinónimo. La siguiente notación nos será útil para las demostraciones: 
\begin{enumerate}
\item Escribiremos $x$ $\in$ $A$ si $x$ es un miembro de $A$, en otro caso escribiremos $x$ $\notin$ $A$.
\item Si $B$ es un subconjunto de $A$, lo denotaremos como $B$ $\subset$ $A$.
\item Si $A \subset B$ y $B \subset A$, entonces $A  = B$.
\item Si $A \subset B$ pero $A \neq B$, diremos que $A$ es un conjunto propio de A.
\item Para todo conjunto $A$, $\empytset$ $\subset$ $A$.
\end{enumerate}
\end{definition}

\section*{Referencia}
Rudin, W., 1987, Real and Complex Analysis, Ed. 3 Cap. 1, Pag. 6 

\section*{Notas Zettelkasten}
\begin{itemize}
  \item \textbf{Enlaces Entrada}: -, -, -, -, -
  \item \textbf{Enlaces Salida}: -, -, -, -, -
  \item \textbf{Inspirado En}: -, --, -
  \item \textbf{Creado A Partir De}: -, -, -, -, -
\end{itemize}
\end{document}