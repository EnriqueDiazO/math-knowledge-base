\documentclass[12pt]{article}
\usepackage{miestilo}
\begin{document}

\section*{Campos}

\begin{definition}

Un \textbf{campo} es un conjunto \( F \) con dos operaciones, llamadas \textit{suma} y \textit{multiplicación}, que satisfacen los siguientes llamados “axiomas de campo” (A), (M) y (D):

\subsection*{(A) Axiomas para la suma}
\begin{itemize}
  \item[(A1)] Si \( x \in F \) y \( y \in F \), entonces su suma \( x + y \in F \).
  \item[(A2)] La suma es conmutativa: \( x + y = y + x \) para todos \( x, y \in F \).
  \item[(A3)] La suma es asociativa: \( (x + y) + z = x + (y + z) \) para todos \( x, y, z \in F \).
  \item[(A4)] \( F \) contiene un elemento \( 0 \) tal que \( 0 + x = x \) para todo \( x \in F \).
  \item[(A5)] A cada \( x \in F \) le corresponde un elemento \( -x \in F \) tal que
  \[
  x + (-x) = 0.
  \]
\end{itemize}

\subsection*{(M) Axiomas para la multiplicación}
\begin{itemize}
  \item[(M1)] Si \( x \in F \) y \( y \in F \), entonces su producto \( xy \in F \).
  \item[(M2)] La multiplicación es conmutativa: \( xy = yx \) para todos \( x, y \in F \).
  \item[(M3)] La multiplicación es asociativa: \( (xy)z = x(yz) \) para todos \( x, y, z \in F \).
  \item[(M4)] \( F \) contiene un elemento \( 1 \ne 0 \) tal que \( 1x = x \) para todo \( x \in F \).
  \item[(M5)] Si \( x \in F \) y \( x \ne 0 \), entonces existe un elemento \( 1/x \in F \) tal que
  \[
  x \cdot (1/x) = 1.
  \]
\end{itemize}

\subsection*{(D) La ley distributiva}
La ley distributiva
\[
x(y + z) = xy + xz
\]
se cumple para todos \( x, y, z \in F \).

\end{definition}

\section*{Referencia}
Rudin, W., 1964, Principles of Mathematical Analysis, Ed. 3 Cap. 1, Pag. 5 

\section*{Notas Zettelkasten}
\begin{itemize}
  \item \textbf{Enlaces Entrada}: -, -, -, -, -
  \item \textbf{Enlaces Salida}: -, -, -, -, -
  \item \textbf{Inspirado En}: -, --, -
  \item \textbf{Creado A Partir De}: -, -, -, -, -
\end{itemize}
\end{document}