\documentclass[12pt]{article}
\usepackage{miestilo}
\begin{document}

\section*{Diferencia de conjuntos}

\begin{definition}
Denotamos con $A \setminus B = \{x : x \in A \notin B \}$ como el complemento de $A$ respecto de $B$. Además, denotamos el complemento de $A$ por $A^{c}$ cuando es claro con respecto a qué conjunto mayor se está tomando el complemento. 
\end{definition}

\section*{Referencia}
Rudin, W., 1987, Real and Complex Analysis, Ed. 3 Cap. 1, Pag. 7 

\section*{Notas Zettelkasten}
\begin{itemize}
  \item \textbf{Enlaces Entrada}: def:conjuntosaaa001, -, -, -
  \item \textbf{Enlaces Salida}: -, -, -, -, -
  \item \textbf{Inspirado En}: -, --, -
  \item \textbf{Creado A Partir De}: -, -, -, -, -
\end{itemize}
\end{document}